\documentclass[a4paper]{scrartcl}

% This helps to minimize margins
\addtolength{\oddsidemargin}{-.875in}
\addtolength{\evensidemargin}{-.875in}
\addtolength{\textwidth}{1.75in}
\addtolength{\topmargin}{-.875in}
\addtolength{\textheight}{1.75in}

\usepackage[utf8]{inputenc}
\usepackage[UKenglish]{babel}
\usepackage{amsfonts}
\usepackage{amsmath}
\renewcommand{\familydefault}{\sfdefault}
\title{Just a Few Notes}
\author{Matthew Horton}
\date{\today}
\begin{document}
\maketitle
\section{Notes and Frequencies}
An octave is a doubling of frequency. Western music traditions split this into twelve geometrically evenly spaced semitones. The ratio between adjacent frequencies is $\sqrt[12]{2} \sim 1.0594630944$. Each note is roughly six percent higher than the last in frequency. A two semi-tone change in frequency is referred to as a tone.

The fifth semitone (known as the fourth in Western music) is $1.3348$ which is very close to $\frac{4}{3}$. 
The seventh semitone (known as the fifth in Western music) is $1.4983$ which is very close to $\frac{3}{2}$. The human ear/brain seems to be able to naturally appreciate these particular intervals.

\section{Major}
In Western music one of the fundamental constructs is the major scale. Having chosen a note, increasing in frequency in the pattern tone, tone, semi-tome, tone, tone, tone will produce a major scale. This consists of eight notes, the final one being double the frequency of the first. 

\section{Intervals}

Major interval is determined by participation in major scale. For some reason some "major" intervals (4th and 5th) are referred to as perfect. There are major and perfect fourths but they require consideration of quarter tones. This table shows intervals by flavour measured by semitones.\\
\begin{tabular}
{| l | c | c | c | c | c | c | c |}
\hline 
Symbol & Flavour\\ \hline
$\circ$ & Diminished  \\ \hline
m & Minor  \\ \hline
M & Major \\ \hline
$P$ & Perfect \\ \hline
$+$ & Augmented \\ \hline
\end{tabular}\\
\begin{tabular}
{| l | c | c | c | c | c | c | c | c | c | c | c | c | c | } \hline
Interval & \multicolumn{11}{c}{Semitones} &  \\ \hline
& 1 & 2 & 3 & 4 & 5 & 6 & 7 & 8 & 9 & 10 & 11 & 12 \\ \hline
2nd & m & M & + & & & & & & & & &    \\ \hline
3rd & & $\circ$ & m & M & + & & & & & & & \\ \hline
4th & & & & $\circ$ & P & + & & & & & & \\ \hline
5th & & & & & & $\circ$ & P & + & & & & \\ \hline
6th & & & & & & & $\circ$ & m & M & + & & \\ \hline
7th & & & & & & & & & $\circ$ & m & M & + \\ \hline
Octave & & & & & & & & & & $\circ$ & m & M \\ \hline
\end{tabular}


\section{Scales}
\begin{tabular}
{| l | r | r | r | c | c | c | c | c | c | c | c | c | } \hline
& Semitones & Intervals & Relative to Major & Slang\\ \hline
Major & 0, 2, 4, 5, 7, 9, 11, 12 & 2, 2, 1, 2, 2, 2, 1 & I, II, III, IV, V, VI, VII, 8\\ \hline
Natural Minor & 0, 2, 3, 5, 7, 8, 10, 12 & 2, 1, 2, 2, 1, 2, 2 & 
1, 2, $\flat$3, 4, 5, $\flat$6, $\flat$7, 8\\ \hline 
Harmonic Minor & 0, 2, 3, 5, 7, 8, 11, 12 & 2, 1, 2, 2, 1, 3, 1 & 
1, 2, $\flat$3, 4, 5, $\flat$6, 7, 8 \\ \hline 
Melodic Minor & 0, 2, 3, 5, 7, 9, 11, 12 & 2, 1, 2, 2, 2, 2, 1 & 
1, 2, $\flat$3, 4, 5, 6, 7, 8 \\ \hline 
+ Pentatonic & 0, 2, 4, 7, 9, 12 & 2, 2, 3, 2, 3 & I, II, III, V, VI, VIII & Drop IV and VIII \\ \hline
- Pentatonic & 0, , 4, 7, 9, 12 & 2, 2, 3, 2, 3 & I, II, III, V, VI, VIII & Drop IV and VIII \\ \hline
\end{tabular}

\begin{tabular}
{| l | r | r | r | c | c | c | c | c | c | c | c | c | } \hline
& Relative to Major & Intervals & Slang\\ \hline
Major &  I, II, III, IV, V, VI, VII, VIII & 2, 2, 1, 2, 2, 2, 1\\ \hline
Natural Minor & I, II, iii, IV, V, vi, vii, VIII\\ \hline 
Harmonic Minor & 0, 2, 3, 5, 7, 8, 11, 12 & 2, 1, 2, 2, 1, 3, 1 & 
1, 2, $\flat$3, 4, 5, $\flat$6, 7, 8 \\ \hline 
Melodic Minor & 0, 2, 3, 5, 7, 9, 11, 12 & 2, 1, 2, 2, 2, 2, 1 & 
1, 2, $\flat$3, 4, 5, 6, 7, 8 \\ \hline 
Major Pentatonic & I, II, III, V, VI, VIII & Drop IV and VIII \\ \hline
Minor Pentatonic & I, iii, IV, V, viii & Drop IV and VIII \\ \hline
Heptatonic Blues & I, iii, IV, V$\flat$, V, viii & Minor Pentatonic add the passing note \\ \hline
\end{tabular}



\section{Chords}
Roman numerals used for the root of the chord. Uppercase is used for major triads and lower case for minor.
\\
\subsection{Triads}
A triad is a three-note chord of stacked thirds.\\
\begin{tabular}
{| l | c | c | c | c | c | c | c | c |} \hline
Flavour & Semitones & Intervals & Notes & Notation & In Key &  \\ \hline
Diminished & (0, 3, 6) & (3, 3) &  (1, 3$\flat$, 5$\flat$) & $^\circ$, dim & c$^\circ$, Cdim & (C, E$\flat$, G$\flat$) \\ \hline
Minor & (0, 3, 7) & (3, 4) & (1, 3$\flat$, 5) & m & c, Cm & (C, E$\flat$, G)  \\ \hline
Major & (0, 4, 7) & (4, 3) & (1, 3, 5) & M & C & (C,E,G) \\ \hline
Augmented & (0, 4, 8) & (4, 4) & (1, 3, 5$\sharp$) & + & C+ & (C,E,G$\sharp$)  \\ \hline
\end{tabular} \\
In popular music dim often refers to to the $7$ chord discussed below.\\
In a major key, there is a mixture of major and minor triads that occur by adhering to the notes of the key. On the leading-tone the triad is diminished. \\
\begin{tabular}
{| l | c | c | c | c | c | c | c |}
\hline 
Root in Key & Chord &  Semitones & Intervals\\ \hline
I - Tonic & I & (0, 4, 7) & (4, 3)  \\ \hline
II - Supertonic & ii & (2, 5, 9) & (3, 4)  \\ \hline
III - Mediant & iii & (4, 7, 11) & (3, 4) \\ \hline
IV - Subdominant & VI & (5, 9, 0) & (4, 3) \\ \hline
V - Dominant & V & (7, 11, 2) & (4, 3) \\ \hline
VI - Submediant & vi & (9, 0, 4) & (3, 4) \\ \hline
VII - Leading-Tone & vii$^\circ$ & (11, 2, 5) & (3, 3) \\ \hline
\end{tabular}

\subsection{Inversion}
When the root note gets moved up an octave this is called the first inversion. The colour of the chord is changed. Sometime this is apparently  notated by a small six to denote that the high pitch note(root) is a sixth above the lowest(third). The second inversion is when the root and third are up an octave and is denoted by a small 4 and 6 stacked. These are different numbers than those used in guitar chords. I don't think you need to worry about this notation.

\subsection{Seventh Chords}
Seventh chords are generally a triad with an added seventh. Formally the chord can be referred by the flavour of the triad and flavour of the seventh. If these are the same the single name is used. The vanilla seventh is the dominant seventh (major minor). This chord sits "naturally" in the key e.g. G$^{7} = $\ G B D F in C.\\
\begin{tabular}
{| l | l | c | c | c | c | c | c | c |} \hline
Flavour & Triad - 7th & Semitones & Intervals & Notes & Notation & In Key \\ \hline
Diminished & Dim dim & (0, 3, 6, 9) & (3, 3, 3) & (1, 3$\flat$, 5$\flat$, 7$\flat\flat$) & $\circ$7, dim7, dim & C$\circ$7, Cdim, Cdim7 \\ \hline
$\frac{1}{2}$-Diminished & Dim Minor & (0, 3, 6, 10) & (3, 3, 4) & (1, 3$\flat$, 5$\flat$, 7$\flat$)  & $\oslash$7, m7-5, m7$\flat$5 & C$\oslash$7, Cm7-5, Cm7$\flat$5 \\ \hline
Minor & Minor Minor & (0, 3, 7, 10) & (3, 4, 3) & (1, 3$\flat$, 5, 7$\flat$) & m7 & Cm7 \\ \hline
Minor major & Minor Major & (0, 3, 7, 11) & (3, 4, 4) & (1, 3$\flat$, 5, 7) & mM7 & CmM7  \\ \hline
Dominant & Major Minor & (0, 4, 7, 10) & (4, 3, 3) & (1, 3, 5, 7$\flat$)  & 7 & C7 \\ \hline
Major & Major Major & (0, 4, 7, 11) & (4, 3, 4) & (1, 3, 5, 7)  & M7, maj7 & CM7, Cmaj7 \\ \hline
Augmented & Aug Minor & (0, 4, 8, 10) & (4, 4, 2) & (1, 3, 5$\sharp$, 7$\flat$) & +7 & C+7 \\ \hline
Augmented Major & Aug Major & (0, 4, 8, 11) & (4, 4, 3) & (1, 3, 5$\sharp$, 7)   & +M7 & C+M7 \\ \hline
\end{tabular} \\
Note the augmented seventh includes a major second (diminished third). \\
\subsection{More Exotic Chords}
\begin{tabular}
{| l | l | c | c | c | c | c | c | c |} \hline
Flavour & Description & Semitones & Intervals & Notes & Notation & In Key \\ \hline
Suspended 9th & Major Triad + 2nd - 3rd & (0, 2, 7) & (2, 5) & (1, 2, 5) & sus2, sus9 & Csus2, Csus9 \\ \hline
Minor Sixth & Minor Triad + 6th & (0, 3, 7, 9) & (3, 4, 2) & (1, 3$\flat$, 5, 6) & m6 & Cm6 \\ \hline
Sixth & Major Triad + 6th & (0, 4, 7, 9) & (4, 3, 2) & (1, 3, 5, 6) & 6 & C6 \\ \hline
Suspended & Major Triad + 4th - 3rd & (0, 5, 7) & (5, 2) & (1, 4, 5) & sus, sus4 & Csus, Csus4 \\ \hline
Fifth & Major Triad - 3rd & (0,7) & (7) & (1, 5) & 5 & C5 \\ \hline

\end{tabular} \\
More....\\
C$^{9} = $\ C$^{7} + 9^{+}$\\
C$^{11} = $\ C$^{7} + 11^{P}$ (I have written no 3rd???) \\

Extended chords apparently include intervening notes so a 13 implies a 7th with the 9th and 11th.

\subsection{Slashes}
X/Y means chord X with Y in the bass. Y may be part of the chord but can also be an added note. For example ensure you play Y with the left hand on piano.


\subsection{Popular Music Chord Notation}
In the key of X ... \\
\begin{tabular}
{| l | l | c | c | c | c | c | c | c |} \hline
Symbol & Alt-Symbols & Chord & Description & Semitones & Intervals & Notes \\ \hline
X & XM or Xmaj & X Major & & & (0,4,7) & (1,3,5) \\ \hline
Xm & & X Minor & & & & (1,3$\flat$,5) \\ \hline
X$^7$ & & X Dominant 7th & & & (0,4,7,10) & (1,3,5,7$\flat$) \\ \hline
X$^5$ & & X Power Chord & & & (0,7,12) & (1,5,8) \\ \hline


\end{tabular} \\



\section{Diatonics}

Triads, \\
\begin{tabular}
{| l | c | c | c | c | c | c | c | c |}
\hline 
Major & I & ii & iii & IV & V & vi & vii$^\circ$  \\ \hline
Natural Minor & i & ii$^\circ$ & III & iv & v & VII & VII   \\ \hline 
Harmonic Minor & i & ii$^\circ$ & III+ & iv & V & VII & vii$^\circ$ \\ \hline
Melodic Minor & i & ii & III+ & VI & V & vi$^\circ$ & vii$^\circ$ \\ \hline
\end{tabular} \\
Sevenths, \\
\begin{tabular}
{| l | c | c | c | c | c | c | c | c |}
\hline 
Major & M7 & m7 & m7 & M7 & 7 & m7 & $\oslash$7  \\ \hline
Natural Minor & m7 & $\oslash$7 & M7 & m7 & m7 & M7 & 7    \\ \hline 
Harmonic Minor & mM7 & $\oslash$7 & +M7 & m7 & 7 & M7 & $\circ$7 \\ \hline
Melodic Minor & mM7 & m7 & +M7 & 7 & 7 & $\oslash$7 & $\oslash$7 \\ \hline
\end{tabular}


\section{Guitar}
When you are in a chord .... find root on low E an place first finger. 1-4, 1-3, 1-3, 1-3, 1-4, 1-4 gives you minor pentatonic. If you put 4 on the root and use the same finger patter you get the major pentatonic

Conisdering the minor pentatonic 1-4, 1-(2)-3, 1-3, 1-3-(4), 1-4, 1-4, these additions are the blue note to make it the heptatonic blues scale. 

Over blues if playing major pentatonic you need to match chord. It will sound awful if you miss chord change.

\section{Harmonica}

\begin{tabular}
{| l | c | c | c | c | c | c | c | c | c | c |}
\hline 
Hole & 1 & 2 & 3 & 4 & 5 & 6 & 7 & 8 & 9 & 10 \\ \hline
Blow & I & III & V & I & III & V & I & III & V & I \\ \hline
Draw & II & V & VII & II & IV & VI & VII & II & IV & VI   \\ \hline 
\end{tabular} \\
The harmonica has a base key. All notes that are available are from the major scale of the key. Blowing gives you contituents of the major chord in this key. Drawing gives you to access to a greater set of notes.


It isn't hard to bend draws.

5-
\subsection{Cross Harp (Second Position)}
Second position is a fifth above the standard key.\\
\begin{tabular}
{| l | c | c | c | c | c | c | c | c | c | c |}
\hline 
Hole & 1 & 2 & 3 & 4 & 5 & 6 & 7 & 8 & 9 & 10 \\ \hline
Blow & IV & VI & I & IV & VI & I & IV & VI & I & IV \\ \hline
Draw & V & I & III & V & vii & II & III & V & vii & II   \\ \hline 
Bent Draw & ? & ? & iii & ? & ? & ? & ? & ? & ? & ?   \\ \hline 
\end{tabular} 
\\
We see the blue scales 2-Draw 3-Bend-Draw 4-Blow 5-Draw 7-Draw 6-Blow

In the key of C\\

\begin{tabular}
{| l | c | c | c | c | c | c | c | c | c | c |}
\hline
Hole & C & E & G & C & E & G & C & E & G & C \\ \hline
Blow & D & G & B & D & F & A & B & D & F & A \\ \hline
\end{tabular} \\


\end{document}
